\chapterauthor{Jay Lofstead}{Sandia National Laboratories}
\chapterauthor{Eric Barton}{Intel}
\chapterauthor{Matthew Curry}{Sandia National Laboratories}
\chapterauthor{Carlos Maltzahn}{University of California, Santa Cruz}
\chapterauthor{Robert Ross}{Argonne National Laboratory}
\chapterauthor{Craig Ulmer}{Sandia National Laboratories}


\chapter{The Convergence of Enterprise, Internet Scale, and High Performance Computing Storage Infrastructures}

\section*{Abstract}
Large scale storage infrastructures have been significantly impacted by the
growth in data analytics applications. High Performance Computing storage
infrastructures, once the extreme end of the storage scale spectrum, must now
adapt to technologies optimized for large scale data analytics applications.
Hardware changes, such as storage class memory, are also affecting how
the exascale storage stack will be constructed. We examine use cases, trends,
convergent technologies, and new opportunities generated by this technology
blending.

\section{Introduction}\label{sec:intro}
this is the intro text


\subsection{A component part}
subsection~\cite{ilyas2004hsn}

%\begin{figure}[htb]
%\begin{figure}[b!]
%\centerline{\includegraphics[width=150pt, height=150pt]{Chapters/chapter1/figures/cat.eps}}
%\caption[List of figure caption goes here]{Figure caption goes here. Figure caption goes here.}
%\end{figure}

%\includegraphics[width=\textwidth]{Chapters/chapter1/figures/cat.eps}
%\caption[Short figure caption]{Figure caption goes here.
%Figure caption goes here.
%Figure caption goes here.}
%\end{figure}

\section{Glossary}
\begin{Glossary}
\item[Adaptable] An adaptable process is designed to maintain effectiveness and
efficiency as requirements change. The process is deemed adaptable when there
is agreement among suppliers, owners, and customers that the process will meet
requirements throughout the strategic period.
\end{Glossary}

